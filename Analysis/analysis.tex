\chapter{Analysis}

\section{Introduction}
	My client is a local church based in central Cambridge, they currently have a pen, paper and verbal system for organising many of the staff, assets and events within the church. There is also no official way to report incidents both pastoral and material.

\subsection{Client Identification}
	My client is a mid-sized city centre church with 250 regular attendees and around 60 people who are involved in rotas and with responsibilities. There are several components to the Church's operation broadly split into two sections: People and resources. People represents all of the meetings, appointments and responsibilities, it also includes records of these meetings and appointments; and resources represents the materials that are required for the smooth operation of the Church, for example Teabags and Support Tickets. The Church also needs to regularly send out emails to inform it's members of their responsibilities and to keep them up to date with church news.

\subsection{Define the current system}
	The current system is a mixture of word of mouth and emails containing scraps of information, people are expected to keep appointments and duties in whatever diary system they have and they do this with a reasonable degree of success. For the cafe stock management, whoever is operating the till keeps a handwritten record of whatever has been sold, which is then added to a central ledger at the end of the day. For all of the incident reporting (support tickets, prayer requests etc) emails are used as the primary way of official communication, often the same set of people are copied into whatever email conversations that take place. For membership inormation, handwritten record cards containing all of the information for each member are stored and archived in two locations.

\subsection{Describe the problems}
%problems:
%	no official reminders of duties and appointments
	The current system does not send out centralized reminders for people who have duties and/or meetings, people are expected to remember their own meetings. This adds potential for people to forget about meetings and duties, especially when they're organised months in advance
%	potential for omissions, limitations in terms of scalability
	The system for dealing with stock management in the cafe adds potential for omissions, the often elderly people responsible for this aspect often forget to write things down and sometimes find it quite difficult to use the old and complex till system. Also all of the orders have to be cross-referenced with a daily price list and any offers or deals are factored in, this adds further potential for error and could mean that the cafe will loose money.
%	no centralised, secure or official record.
%	no synchronisation/potential for omissions, data protection act states that only appropriate and up to date information is stored - these record cards do not meet these standards.

\subsection{Section appendix}

\section{Investigation}

\subsection{The current system}

\subsubsection{Data sources and destinations}

\subsubsection{Algorithms}

\subsubsection{Data flow diagram}

\subsubsection{Input Forms, Output Forms, Report Formats}

\subsection{The proposed system}

\subsubsection{Data sources and destinations}

\subsubsection{Data flow diagram}

\subsubsection{Data dictionary}

\subsubsection{Volumetrics}

\section{Objectives}

\subsection{General Objectives}

\subsection{Specific Objectives}

\subsection{Core Objectives}

\subsection{Other Objectives}

\section{ER Diagrams and Descriptions}

\subsection{ER Diagram}

\subsection{Entity Descriptions}

\section{Object Analysis}

\subsection{Object Listing}

\subsection{Relationship diagrams}

\subsection{Class definitions}

\section{Other Abstractions and Graphs}

\section{Constraints}

\subsection{Hardware}

\subsection{Software}

\subsection{Time}

\subsection{User Knowledge}

\subsection{Access restrictions}

\section{Limitations}

\subsection{Areas which will not be included in computerisation}

\subsection{Areas considered for future computerisation}

\section{Solutions}

\subsection{Alternative solutions}

\subsection{Justification of chosen solution}
