\chapter{Analysis}

\section{Introduction}

\section{The Client & Users}
The client is the physics department of Long Road Sixth Form College, who have requested software to demontsrate the core concepts within the new specification AS Physics. The client has requested that the software can be used by both the students and the teachers both inside and outside of class, therfore, the software should work on college PCs without any additional resources (assuming the PC has python3, OpenGL and PyQt installed, which is standard installation in Long Road Sixth Form). The simulation and rendering needs to work in real time (or faster) and it needs to be easy to use.

\section{Specific Requirements}
The physics that the simulator is capable of demonstrating should match the Mechanics section of the new specification physics AS, the specific list of physical phenomena that should be simulated are: 
\begin{itemize}
	\item Gravity  - objects should be subject to gravity
	\item Collisions
	\begin{itemize}
		\item Basic collisions - Objects should not overlap
		\item Bounce - Objects should bounce (see momentum)
	\end{itemize}
	\item Kinematics - Objects should use the basic rules of kinematics to determine their movement
	\item Forces - Objects should exert forces on eachother, and react to forces and impulse
	\item Centre of Gravity and Moment - Objects should have a centre of gravity, and they should have moments.
\end{itemize}

The simulator program should also have the following utility features:
\begin{itemize}
	\item Save/Load Simulations - The program should be able to save and load simulations at a certain state
	\item The ability to change the time scale of the simulation.
	\item The software needs to execute in approximately real time (24fps)
	\item The software needs some way of representing the pysical object as a text file. (JSON)

\end{itemize}


\section{Summary}
In short, the system that I propose will add another degree of interactivity to the AS course, and it'll accurately demonstrate a conprehensive set of physical laws and relationships that will be covered in the new specification AS. The software will be a freely scriptable physics engine, rather than a fixed set of customiseable simulations - it could maybe even be developed into a 2d game engine, this means that the user will be able to create a 2d world from scratch and use it to demonstrate, experiment, or just play with any physical object that they can define.