\chapter{Analysis}

\section{Introduction}

\section{The Client and Users}
The client is the physics department of Long Road Sixth Form College, who have requested software to demontsrate the core concepts within the new specification AS Physics. The client has requested that the software can be used by both the students and the teachers both inside and outside of class, therfore, the software should work on college PCs without any additional resources (assuming the PC has python3, OpenGL and PyQt installed, which is standard installation in Long Road Sixth Form). The simulation and rendering needs to work in real time (or faster) and it needs to be easy to use.

\section{Specific Requirements}
The specific requirements for this simulator can be derived from the Physics AS new specification (2015), a list of which are:
\begin{itemize}
	\item Gravity  - objects should be subject to gravity.
	\item Collisions
	\begin{itemize}
		\item Basic collisions - Objects should not overlap.
		\item Bounce - Objects should bounce (see momentum).
	\end{itemize}
	\item Kinematics - Objects should use the basic rules of kinematics to determine their movement.
	\item Forces - Objects should exert forces on eachother, and react to forces and impulse.
	\item Centre of Gravity and Moment - Objects should have a centre of gravity, and they should have moments.
	\item Momentum and Impulse - Objects should simulate the effects of momentum and impulse according to Newton's laws.
\end{itemize}

The simulator program should also have the following utility features:
\begin{itemize}
	\item Save/Load Simulations - The program should be able to save and load simulations at a certain state.
	\item The ability to change the time scale of the simulation.
	\item The software needs to execute in approximately real time (at least 15fps).
	\item The software needs some way of representing the pysical object as a text file. (JSON)

\end{itemize}

\section{Current System}

	\subsection{Algorithms}
		The current system uses a series of physical equations and formulae to find the values of physical quantities at a certain time - they're always expressed in terms of \textit{t}(time) and they all can be derived to find the position of a physical body. So the algorithms in are for changing the official SI equations and formulae algebreically so that they're in terms of \textit{t}.
	\subsection{Data Storage}
		The current system does not store the data from these equations peristantly.


\section{Processing Requirements}
One of the key requirements for this system is that the simulation runs and renders smoothly, at approximately real time, this means that it is of paramount importance that the code is written in such a way that it executes quickly, and that it manages the host PC's resources efficiently. All of the college PCs have multicore processors, but they do not have dedicated GPUs which would ease the CPU strain for real-time physics simulation. This means that the system will have to use multithreading to accelerate the simulation and the rendering.

If all else fails, and the simulator doesn't render in real time, there should be an option to reduce the output resolution scaling, which should dramatically reduce the processing required to simulate and render.

\section{Data Flow}
\subsection{Input}
	The user will input some or all of the physical values into an input form. The simulation is designed to be open, so that the user can enter as many or as few properties as they want.

\subsection{Storage}
	The user's input will be stored in table within a database file. That data will then be used to generate an initial frame, which is then in turn stored in the same database. As each frame is generated, it'll store it in this same database file. This means that there are three 'frames' stored in the database: A pre-initial frame, which has the raw data that the user input, an initial frame which contains the generated data for the first frame; and the latest frame simulated. The following path will be followed: Render - Simulate - Store - Render - Simulate etc.

\subsection{Database}

\subsubsection{Entity Relationship Diagrams}
	Here is where I'll put the entity relationship diagrams

\subsubsection{Data flow diagrams}
	DFDs


\subsection{Output}
	The simulation data will be  converted into an updated serires of shapes and positions which will be drawn to an OpenGL window. To the user, this should appear to be a stable, flowing animation of the shapes moving, rather than a set of individual frames.

\section{Possible Methods of Solution}
	There are many existing physics frameworks and they the majority of them work using the same basic mathematical functions to get the change in state of an object. A simplified version of these equations is required for deomonstrations at AS level, so for this software I would have to write bespoke versions of the existing frameworks.

	As far as rendering the simulation is concerned, OpenGL will have all the low-level code already covered, all I have to do is set up a render pipeline and get it to animate, this means that I won't have to deal with OS or platform specific system requirements hence my application will be portable enough to work on most, if not all College PCs.


\section{Summary}
In short, the system that I propose will add another degree of interactivity to the AS course, and it'll accurately demonstrate a conprehensive set of physical laws and relationships that will be covered in the new specification AS. The software will be a freely scriptable physics engine, rather than a fixed set of customiseable simulations - it could maybe even be developed into a 2d game engine, this means that the user will be able to create a 2d world from scratch and use it to demonstrate, experiment, or just play with any physical object that they can define.
