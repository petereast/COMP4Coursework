\chapter{Analysis}

\section{Introduction}
	My client is a local church based in central Cambridge, they currently have a pen, paper and verbal system for organising many of the staff, assets and events within the church. There is also no official way to report incidents both pastoral and material.

\subsection{Client Identification}
	My client is a mid-sized city centre church with 250 regular attendees and around 60 people who are involved in rotas and with responsibilities. There are several components to the Church's operation broadly split into two sections: People and resources. People represents all of the meetings, appointments and responsibilities, it also includes records of these meetings and appointments; and resources represents the materials that are required for the smooth operation of the Church, for example Teabags and Support Tickets. The Church also needs to regularly send out emails to inform it's members of their responsibilities and to keep them up to date with church news.

	They use a mixture of Apple iMac and Windows machines, so it's essential that any software they use is cross-compatible and doesn't use any OS specific features while also supporting all the quirks within a particular OS. The church also has a central server and networking capabilities which they use for filesharing and remote access of devices such as projectors.

	The Church relies heavily on being properly organised, which places a large amount of responsibility and pressure on the office administrator and staff team. Many of these tasks can be automated, however there are no automatic systems in place. Also because the Church leadership is elected and the people holding the various roles change regularly, ofteh confusion emerges during transition periods.

\subsection{Define the current system}
	The current system is a mixture of word of mouth and emails containing scraps of information. It consists of three main elements: the meetings, duties and appointments part of the system, the support tickets and referrals system and the stock management system (this includes cafe resources, office stationary and anything that has to be purcheased regularly or has a supply which can be depleted). Currently, people are expected to keep appointments and duties in whatever diary system they have and they do this with a reasonable degree of success. For the cafe stock management, whoever is operating the till keeps a handwritten record of whatever has been sold, which is then added to a central ledger at the end of the day. For all of the incident reporting (support tickets, prayer requests etc) emails are used as the primary way of official communication, often the same set of people are copied into whatever email conversations that take place. For membership inormation, handwritten record cards containing all of the information for each member are stored and archived in two locations.

\subsection{Describe the problems}
%problems:
%	no official reminders of duties and appointments
	The current system does not send out centralized reminders for people who have duties and/or meetings, people are expected to remember their own meetings. This adds potential for people to forget about meetings and duties, especially when they're organised months in advance
%	potential for omissions, limitations in terms of scalability
	The system for dealing with stock management in the cafe adds potential for omissions, the often elderly people responsible for this aspect often forget to write things down and sometimes find it quite difficult to use the old and complex till system. Also all of the orders have to be cross-referenced with a daily price list and any offers or deals are factored in, this adds further potential for error and could mean that the cafe will loose money. The lack of comprehensive records means that the cafe will have trouble accurately guaging trends in the customers and fail to adapt to these trends.
%	no centralised, secure or official record.
	Because a lot of the pastoral information that has to be communicated to people in various positions around the church is strictly confidential, email is not a bad choice, however it does not provide a centralised set of records which would be required to ensure professional and caring conduct. So the current system fails to ensure that if any official communication is called into question that there is an official, untampered record of it.
%	no synchronisation/potential for omissions, data protection act states that only appropriate and up to date information is stored - these record cards do not meet these standards.
	The data protection act states that information about anyome must not be kept any longer than necessary, and it must be kept up to date, some of these record cards date back to the 1920s and contain information about people who no longer attend the church and people who are deceased, there is no easy way to update these record cards and there are no failsafes the ensure their validity and their complience with the DPA.

\subsection{Section appendix}

\section{Investigation}

\subsection{The current system}

\subsubsection{Data sources and destinations}
Because the Church relies on information being shared between different people with different responsibilities, this information has to move between people.

\begin{table}[]
\centering
\caption{My caption}
\label{my-label}
\begin{tabular}{llll}
\hline
\multicolumn{1}{|l|}{Source}             & \multicolumn{1}{l|}{Data}                     & \multicolumn{1}{l|}{Example}                                                                         & \multicolumn{1}{l|}{Destination}                       \\ \hline
\multicolumn{1}{|l|}{Member}             & \multicolumn{1}{l|}{Name and Contact details} & \multicolumn{1}{l|}{John Smith, jsmith@mailmailmail.com}                                             & \multicolumn{1}{l|}{Membership Records}                \\ \hline
\multicolumn{1}{|l|}{Appointment}        & \multicolumn{1}{l|}{Who, what when and where} & \multicolumn{1}{l|}{John smith, pastoral visit, 10:30 Thursday  4th Oct 2013, Church Meeting room 4} & \multicolumn{1}{l|}{Attendee's Diaries}                \\ \hline
\multicolumn{1}{|l|}{Referral Ticket}    & \multicolumn{1}{l|}{Who and what}             & \multicolumn{1}{l|}{John Smith, Prayer Request for his niece's hospital visit}                       & \multicolumn{1}{l|}{Prayer Team's records}             \\ \hline
\multicolumn{1}{|l|}{Support ticket}     & \multicolumn{1}{l|}{What and when}            & \multicolumn{1}{l|}{Faulty Microphone, Sunday xth of January.}                                       & \multicolumn{1}{l|}{Finance Team's 'to purchase' list} \\ \hline
\multicolumn{1}{|l|}{Cafe Order}         & \multicolumn{1}{l|}{What, when and how much}  & \multicolumn{1}{l|}{2x cup of tea, Thursday 4th Oct 2015, £3.65}                                     & \multicolumn{1}{l|}{Cafe Management's ledger}          \\ \hline
\multicolumn{1}{|l|}{Resources Purchase} & \multicolumn{1}{l|}{What, when and how much}  & \multicolumn{1}{l|}{3x replacement microphone, 4x AA batteries; Wednesday 3rd Oct 2015, £215.30}     & \multicolumn{1}{l|}{Finance Team's ledger}             \\ \hline
\multicolumn{1}{|l|}{}                   & \multicolumn{1}{l|}{}                         & \multicolumn{1}{l|}{}                                                                                & \multicolumn{1}{l|}{}                                  \\ \hline
                                         &                                               &                                                                                                      &                                                        \\
                                         &                                               &                                                                                                      &                                                        \\
                                         &                                               &                                                                                                      &
\end{tabular}
\end{table}


\subsubsection{Algorithms}
 The current system uses few algorithms for it's operations, for the current system doesn't actually do anything, it simply tells people to do things.
For Pastoral or Support tickets:
\begin{algorithm}[H]
	\caption{Support Referral request:}
	\begin{algorithmic}[1]
		\RECEIVE{$The problem$}
		\Repeat
			\RECEIVE{$Is the problem solved?$}
		\Until{$Problem is solved$}
	\end{algorithmic}
\end{algorithm}

And for the resources stock management:
\begin{algorithm}[H]
	\caption{Stock management algorithm}
	\begin{algorithmic}[1]
		\RECEIVE{$The minimum safe amount of this resource$}
		\RECEIVE{$The cost of this item$}
		\RECEIVE{$The current amount of this resource$}
		\IF{$The current amount of the resources \textless minumum required amount of this resource$}
			\SEND{$Buy more of this resource$}
	\end{algorithmic}
\end{algorithm}

\subsubsection{Data flow diagram}

\subsubsection{Input Forms, Output Forms, Report Formats}

\subsection{The proposed system}

\subsubsection{Data sources and destinations}

\begin{table}[]
\centering
\caption{My caption}
\label{my-label}
\begin{tabular}{llll}
Source                  & Data                           & Example                                                                & Destination                         \\
Member                  & Name, Contact info, other info & John Smith, jsmith@smithmail.com                                       & Member's database                   \\
Appointment/Meeting     & Where, when and who            & John Smith, Meeting room 4, Thurstay 4th Oct                           & Appointments database               \\
Apointment entity       & Where, when and who            & John Smith and James East, jsmith@smithmail.com                        & Personal Appointments list          \\
Item                    & What, Cost                     & Sennheiser ew300 fn3 Microphone, £800                                  & Items Database                      \\
Purchase Request        & Items, Quantity, Cost, When    & \{Sennheiser ew300 fn3 Microphone, £800\}, 5, £4000, Wednesday 3rd Oct & Purchase Request database           \\
Support Ticket          & What, when                     & \{item\}, Tuesday 2nd Oct                                              & Prioritized list of support tickets \\
List of support tickets & What, when                     & \{item\}, Tuesday 2nd Oct                                              & Personal Task list                  \\
Personal Task list      & What, details, deadline        & Support Ticket, \{ticket\}, Wednesday 3rd Oct                          & Weekly adgenda                      \\
Cafe Stock item         & What, Cost                     & Teabags, £0.01                                                         & Cafe Resource                       \\
Cafe Resource           & \{cafe stock item\}, quantity  & \{teabags, £0.01\}, 50                                                 & Cafe Resources List
\end{tabular}
\end{table}

\subsubsection{Data flow diagram}

\subsubsection{Volumetrics}
	The church has between 150 and 250 congregates meeting every week, of these congregates there are about 50 'members'. In the Baptist Church, a member is defined as someone who attends regularly enough and is commited to both Christianity and the Church who is able to make proposals and cast votes in the monthly Church administration meeting. This means that the scale of the databases required for staff management will not be huge. However, for the meetings management elements of the system, there is not a limit to the number of meetings and appointments that will be organised therfore this part of the system must be scaleable, assuming


\section{Objectives}

\subsection{General Objectives}

\subsection{Specific Objectives}

\subsection{Core Objectives}

\subsection{Other Objectives}

\section{ER Diagrams and Descriptions}

\subsection{ER Diagram}

\subsection{Entity Descriptions}

\subsection{Data dictionary}

\section{Object Analysis}

\subsection{Object Listing}

\subsection{Relationship diagrams}

\subsection{Class definitions}

\section{Other Abstractions and Graphs}

\section{Constraints}

\subsection{Hardware}

\subsection{Software}

\subsection{Time}

\subsection{User Knowledge}

\subsection{Access restrictions}

\section{Limitations}

\subsection{Areas which will not be included in computerisation}

\subsection{Areas considered for future computerisation}

\section{Solutions}

\subsection{Alternative solutions}

\subsection{Justification of chosen solution}
